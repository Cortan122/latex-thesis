\section{Назначение программы}
\subsection{Наименование программы на русском языке}

Разработка расширений компилятора C для поддержания процедурно-параметрического программирования.

\subsection{Наименование программы на английском языке}

Development of C Compiler Extensions for the Procedural-Parametric Paradigm.


\subsection{Область применения}
Расширенный компилятор языка С может быть использован в различных сферах, где требуется быстрое расширение и постоянная поддержка программ, но использование Объектно-Ориентированного подхода недопустимо.
Например, использование ООП на встроенных устройствах может быть затруднено из-за ограничений на память и процессорные ресурсы.

Основное назначение компилятора --- предоставление инструмента для использования проце\-дурно-параметрических языков программирования, позволяя программистам легко определять новые типы и мультиметоды, ускоряя процесс разработки и повышая качество программного кода.


\section{Условия выполнения программы}
{
  \renewcommand{\subsubsection}[1]{\subsection{#1}}
  \subsubsection{Климатические условия}
Климатические условия совпадают с климатическими условиями эксплуатации устройства.

\subsubsection{Требования к пользователю}
Пользователь должен иметь базовое представление об основах использования персональных на базе операционной системы Linux.

Также от пользователя требуется:
\begin{enumerate}
  \item Среднее школьное образование.
  \item Способность механически взаимодействовать с персональным компьютером и запускать программу.
  \item Знание синтаксиса языка программирования C, который будет компилироваться.
  \item Понимание основных понятий компиляции и выполнения программ.
  \item Умение работать с командной строкой для запуска компилятора и запуска скомпилированных программ.
\end{enumerate}

\subsection{Требования к составу и параметрам технических средств}
Для успешной установки программы необходим 1 МБ свободной памяти на диске устройства.
Для корректной работы компилятора необходимо 10 МБ оперативной памяти.

\subsection{Требования к информационной и программной совместимости}
Для корректной работы программы необходима Unix-подобная операционная система с установленным компилятором GCC.

}

\section{Выполнение программы}
\subsection{Установка программы}
В комплект поставки программы входит архив с технической документацией, готовой реализацией программного продукта и презентацией проекта.
Перед проведением испытаний необходимо распаковать архив с исходным кодом, открыть окно терминала с командной оболочкой Bash, и скомпилировать программу, используя команду \cmd{make all}.
% Для установки программы необходимо загрузить на компьютер архив CRTzip и распаковать его.

\subsection{Запуск программы}
Для начала работы с программой необходимо открыть окно терминала в корневой директории исходного кода программы, в которую была произведена её компиляции.
Взаимодействия с метакомпилятором проводятся через два исполнимых файла:
\begin{enumerate}
  \item Файл \cmd{./main} предоставят возможность конвертировать уже препроцессированный код на расширенном языке программирования C в стандартный код.
  Имя входного файла передаётся первым аргументом, и допускается использование исходного непрепроцессированного кода, хотя этот функционал не рекомендован к использованию.
  Также предоставляется опция перенаправления выхода в отдельный файл, имя которого передаётся вторым аргументом в команду.
  В таком случае в поток стандартного выхода выводиться список определений, который метакомпилятор нашёл в данной ему программе. Файл \cmd{./main} не обрабатывает командные опции.

  \item Файл \cmd{./wrapper} предоставляет командный интерфейс, схожий с любым другим компилятором языка C, разработанным под Unix подобную систему.
  Он принимает на вход, в виде командного аргумента, имя файла с исходном кодом на расширенном языке программирования C, запускает на этом коде препроцессор, метакомпилятор и компилятор, и организовывает межпроцессорные потоки для передачи данных.
  Также все командные опции, то есть аргументы начинающиеся с символа \cmd{-}, передаются запущенному компилятору.
  Таким образом можно, например, указать имя создаваемого исполняемого файла.
\end{enumerate}

\subsection{Работа с метакомпилятором}
Перед началом работы с метакомпилятором необходимо написать программу на расширенном языке C, который более подробно описан в пункте \ref{sec:howtowrite}, и сохранить её исходный код в файл с расширением \cmd{.c}.

% todo: стиль
После того как вы создали программу, её необходимо скомпилировать. Это можно сделать двумя разными способами:
\begin{enumerate}
  \item Использовать готовый исполняемый файл \cmd{./wrapper}, который сам проводит все действия необходимые для успешной компиляции.
  Этот простой подход идеально подходит для большинства простых ситуаций, но менее гибкий чем его альтернатива.

  \item В ручную провести все три шага компиляции, а именно препроцессирование, метакомпиляцию и компиляцию, используя исполнимый файл \cmd{./main}. Их можно объединить в одну команду: \cmd{gcc -E файл.c -o - | ./main /dev/stdin | gcc -x c -}.

  Также такой подход предоставит вам возможность передавать разные опции препроцессору и самому компилятору.
  В случаях возникновения каскадных синтаксических ошибок, стоит запустить команду \cmd{./main файл.c} без препроцессора и компилятора, чтобы было проще понять точный характер ошибки, и исправить её.
\end{enumerate}

Также в комплекте с метакомпилятором имеется некоторое количество примеров, программ, написанных на расширенном языке.
Например, можно запустить скриптовый файл \cmd{./build_example.sh}, который соберёт многофайловую программу, исходный код которой находиться в директории \cmd{example} \CRTfigref{buildexamplesh}{Использование скрипта \cmd{./build_example.sh}}.

\subsubsection{Примеры конкретных использований}
В данном пункте приведены примеры команд, которые могут быть полезны в некоторых распространённых случаях.
\begin{enumerate}
  \item Изменение имени получаемого исполняемого файла.

  {\centering\fbox{\cmd{./wrapper -omyexe файл.c}}}

  \item Использование санитизации адресов (Address Sanitizer).

  {\centering\fbox{\cmd{./wrapper -fsanitize=address файл.c}}}

  \item Компиляция в объектный файл, для дальнейшей интеграции в более сложную программу.

  {\centering\fbox{\cmd{./wrapper -c -oфайл.o файл.c}}}

  В дальнейшем такой файл можно будет использовать обычный компоновщик GCC для объединения файлов в одну программу \CRTfigref{multifiledemo}{Компиляция разделённой программы}.

  \item Использование внешней библиотеки, через механизм динамического компоновщика системы.

  {\centering\fbox{\cmd{./wrapper -lpng файл.c}}}

\end{enumerate}

\subsection{Написание программ}
\label{sec:howtowrite}
Расширенный язык представляет собой стандартный язык программирования C с некоторым количеством изменений, которые описаны в следующей грамматике:

\begin{grammar}
  <ОбобщеннаяСтруктура> ::= <Структура> `<' <СписокСпециализаций> `>'

  <СписокСпециализаций> ::= (<ИмяТипа> `;')*
\end{grammar}

\begin{grammar}
  <РасширениеТипа> ::= <ИмяСтруктуры> `+' `<' <СписокСпециализаций> `>'
\end{grammar}

\begin{grammar}
  <ПараметрическаяФункция> ::= <ТипВозврата> <ИмяФункции> \\
  `<' <СписокПараметрическихАргументов> `>' `(' <СписокАргументов> `)' \\
  (<ТелоФункции> | `=' `0')

  <СписокПараметрическихАргументов> ::= <Аргумент> (`,' <Аргумент>)*

  <Аргумент> ::= <ОбобщённыйТип> `*' <ИмяАргумента>
\end{grammar}

\begin{grammar}
  <СпециализированнаяФункция> ::= <ТипВозврата> <ИмяФункции> \\
  `<' <СписокСпецПараметров> `>' `(' <СписокПараметров> `)'
  <ТелоФункции>

  <СписокСпецПараметров> ::= <СпецАргумент> (`,' <СпецАргумент>)*

  <СпецАргумент> ::= <ОбобщённыйТип> `<' <ИмяТипа> `>' `*' <ИмяАргумента>
\end{grammar}

\begin{grammar}
  <ВызовПараметрическойФункции> ::= <ИмяФункции> \\
  `<' <СписокОбобщённыхПараметров> `>' `(' <СписокПараметров> `)'
\end{grammar}

\begin{grammar}
  <Специализация> ::= <ИмяСтруктуры> `<' <ИмяТипа> `>'

  <СпецИнициализатор> ::= <Инициализатор> `<' <Инициализатор> `>'
\end{grammar}

\begin{grammar}
  <ДоступПоля> ::= <ИмяПеременной> `!' <ИмяПоля> | \\
  <ИмяУказателя> `->!' <ИмяПоля>
\end{grammar}

Эти правила продукции определяют грамматику расширенного языка, подобного C, и позволяют использовать обобщенные структуры, параметрические функции, расширения типов, специализации и доступ к полям.
Они также определяют специальные операторы \cmd{->!} и \cmd{!} для доступа к полям обобщённых структур и расширенный синтаксис инициализации структур.
