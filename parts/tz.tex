\section{Введение}
\subsection{Наименование программы на русском языке}

Разработка расширений компилятора C для поддержания процедурно-параметрического программирования.

\subsection{Наименование программы на английском языке}

Development of C Compiler Extensions for the Procedural-Parametric Paradigm.


\subsection{Краткая характеристика области применения}
Расширенный компилятор языка С может быть использован в различных сферах, где требуется быстрое расширение и постоянная поддержка программ, но использование Объектно-Ориентированного подхода недопустимо.
Например, использование ООП на встроенных устройствах может быть затруднено из-за ограничений на память и процессорные ресурсы.

Основное назначение компилятора --- предоставление инструмента для использования проце\-дурно-параметрических языков программирования, позволяя программистам легко определять новые типы и мультиметоды, ускоряя процесс разработки и повышая качество программного кода.


% Приложение для мобильного устройств Android предоставляющее доступ к материалам базы знаний по психологии, может быть использовано как в учебных заведениях, так и для индивидуального самообразования и самопомощи.

% Основное назначение приложения --- предоставление инструмента для изучения психологии на мобильных устройствах Android. Приложение должно предоставлять доступ к материалам, таким как курсы и опросники, а также статьи, рекомендации, техники и факты, для повышения уровня знаний в психологии и улучшения психического здоровья.

\subsection{Основание для разработки}
Приказ декана факультета компьютерных наук И.В. Аржанцева о закреплении тем ВКР за студентами и назначении руководителей и консультантов ВКР № 2.3-02/151222-7 от декабря 2022 года.

% Приказ декана факультета компьютерных наук Национального исследовательского университета <<Высшая школа экономики>> № 2.3-02/1212-01 от 12.12.2017 <<Об утверждении тем, руководителей курсовых работ студентов образовательной программы Программная инженерия факультета компьютерных наук>>.


\section{Назначение разработки}
\subsection{Функциональное назначение}

Функциональным назначением проекта является расширение компилятора для поддержки процедурно-параметрического подхода и создание новых языковых конструкций для удобства программирования в этом подходе.

% Функциональным назначением приложения является обеспечение процесса ознакомления с материалом из базы знаний по психологии. Ознакомление может быть как с текстовым материалом, так и прохождением курсов или опросов.

% К текстовым материалам относятся:

% \begin{enumerate}
%   \item Рекомендации --- книги, фильмы и сериалы, которые помогут улучшить понимание себя и окружающего мира;
%   \item Факты --- психологические факты и термины;
%   \item Техники --- медитации, упражнения на управление стрессом, упражнения на улучшение настроения и многое другое;
%   \item Статьи --- различного рода текстовые материалы на тему мотивация, эмоций, личностного роста, управления стрессом и многое другое;
% \end{enumerate}

% Опросники позволят пользователям лучше понять себя и свои эмоции. Они могут охватывать такие темы, как личностные черты, управление стрессом, межличностные отношения и многое другое. В конце каждого опросника пользователь получит результат на основе своих ответов. Курсы, в свою очередь, позволяют изучить уроки с теоретическим и практическим материалом.

\subsection{Эксплуатационное назначение}

Разрабатываемый компилятор может быть полезен в областях, где необходимо расширять типы данных на основе уже существующих. Он может использоваться в программировании приложений с большим объёмом кода, где типы данных не всегда являются очевидными и описываются только в процессе разработки.

В данном случае расширенный язык позволит создавать новые типы данных на основе уже существующих, что упростит процесс разработки и избавит разработчиков от лишней работы. Это может быть особенно полезно в крупных проектах, где большой объём кода требует постоянного обновления и поддержки. Также он позволит сократить время на написание новых типов данных и повысить производительность программ.

Пользователями нашего проекта могут быть программисты и разработчики, работающие с языком C и заинтересованные в расширении его возможностей. Это может включать как студентов, изучающих язык C в академической среде, так и профессионалов, использующих его в коммерческих проектах.


\section{Требования к программе}
\subsection{Требования к функциональным характеристикам}
Разрабатываемый программный продукт должен:

\begin{enumerate}
  \item Предоставлять поддержку расширенных конструкций языка, основанных на принципах Процедурно-Параметрического программирования~\cite{легалов2016эволюционное}, а именно:
  \begin{enumerate}
    \item Определение параметрического обобщения;
    \begin{grammar}
      <ОбобщеннаяСтруктура> ::= <Структура> `<' <СписокСпециализаций> `>'

      <СписокСпециализаций> ::= <ИмяТипа> `;' | <ИмяТипа> `;' <СписокСпециализаций>
    \end{grammar}
    \item Расширение обобщённой структуры;
    \begin{grammar}
      <РасширениеТипа> ::= <ИмяСтруктуры> `+' `<' <СписокСпециализаций> `>'
    \end{grammar}
    \item Использование конкретной специализации, в качестве типа переменных;
    \begin{grammar}
      <Специализация> ::= <ИмяСтруктуры> `<' <ИмяТипа> `>'
    \end{grammar}
    \item Определение специализированной функции;
    \begin{grammar}
      <СпециализированнаяФункция> ::= <ТипВозврата> <ИмяФункции> \\ `<' <СписокОбобщённыхПараметров> `>' `(' <СписокПараметров> `)'
    \end{grammar}
    \item Вызов специализированной функции;
    \begin{grammar}
      <ВызовПараметрическойФункции> ::= <ИмяФункции> \\ `<' <СписокОбобщённыхПараметров> `>' `(' <СписокПараметров> `)'
    \end{grammar}
  \end{enumerate}

  \item Предоставлять поддержку базовых конструкций языка программирования C, таких как переменные, функции, условные операторы и циклы, пропуская код без изменений;
  \item Предоставлять возможность объявления пользовательских типов данных и структур данных;
  \item Предоставлять возможность компиляции одной программы используя несколько единиц компиляции;
  \item Предоставлять возможность обработки исходного кода на языке C, после того как он прошёл этап препроцессора;
  \item Предоставлять возможность получения на выходе кода с правильными номерами строк, которые контролируются особыми директивами препроцессора;

  \item Запускать компилятор и препроцессор языка C и организовывать межпроцессную коммуникацию между ними;
  \item Хранить таблицу имён переопределённых типов, обобщённых структур, специализированных переменных и параметрических функций.
\end{enumerate}


\subsection{Требования к интерфейсу}
Интерфейс должен соответствовать следующим требованиям:
\begin{enumerate}
  \item Интерфейс должен быть реализован через командную строку, в соответствии с конвенциями Unix-подобных операционных систем.
  \item Возможность указания входного файла с исходным кодом и выходного файла с скомпилированным кодом.
  \item Поддержка различных опций, таких как указание уровня оптимизации, использование определенной архитектуры процессора, указание директории с заголовочными файлами. Это опции в дальнейшем будут передаваться компилятору стандартного языка C.
  \item Вывод информации о процессе компиляции, такой как сообщения об ошибках и предупреждениях.
\end{enumerate}


\subsection{Требования к формату входных и выходных данных}

\subsubsection{Входные данные}

Компилятор принимает на вход исходный код на расширенном языке программирования C из файла или временного потока.
Эти данные потом опционально передаются стандартному препроцессору языка C, установленному на системе.

\subsubsection{Выходные данные}

Выходными данными программы является исходный код на стандартном языке программирования C, где все расширенные конструкции заменены на их эквиваленты в стандартном C.
Также предоставляется опции передать изменённый исходный код сразу на следующий этап компиляции.

\subsection{Требования к надежности}

\begin{enumerate}
  \item Программа не должна препятствовать устойчивому функционированию других программных продуктов на устройстве пользователя;
  \item При синтаксических ошибках в исходном коде, программа должна сообщать пользователю об их наличии и не завершаться аварийно;
  \item В программе должны отсутствовать утечки памяти и ошибки сегментации;
  \item Программа должна корректно осуществлять свою работу при любом вводе данных пользователя.
\end{enumerate}


\subsection{Условия эксплуатации}
\subsubsection{Климатические условия}
Климатические условия совпадают с климатическими условиями эксплуатации устройства.

\subsubsection{Требования к пользователю}
Пользователь должен иметь базовое представление об основах использования персональных на базе операционной системы Linux.

Также от пользователя требуется:
\begin{enumerate}
  \item Среднее школьное образование.
  \item Способность механически взаимодействовать с персональным компьютером и запускать программу.
  \item Знание синтаксиса языка программирования C, который будет компилироваться.
  \item Понимание основных понятий компиляции и выполнения программ.
  \item Умение работать с командной строкой для запуска компилятора и запуска скомпилированных программ.
\end{enumerate}

\subsection{Требования к составу и параметрам технических средств}
Для успешной установки программы необходим 1 МБ свободной памяти на диске устройства.
Для корректной работы компилятора необходимо 10 МБ оперативной памяти.

\subsection{Требования к информационной и программной совместимости}
Для корректной работы программы необходима Unix-подобная операционная система с установленным компилятором GCC.


\subsection{Требования к программной документации}
\ifsmallfile\else\label{sec:doclist}\fi
\ifsmallfile\label{sec:doclist}\fi
В рамках данной работы должна быть разработана следующая программная документация в соответствии и ГОСТ ЕСПД:
\begin{itemize}
  \item Техническое задание \cite{gostTZ};
  \item Программа и методика испытаний \cite{gostPMI};
  \item Текст программы \cite{gostTP};
  \item Руководство оператора \cite{gostRO};
  \item Текст выпускной квалификационной работы (ВКР).
\end{itemize}

Текст ВКР также обязан содержать все остальные документы в качестве приложений и иметь сквозную нумерацию страниц.

\subsection{Специальные требования к программной документации}
Документы к программе должны быть выполнены в соответствии с ГОСТ 19.106-78 и ГОСТами к каждому виду документа
(см. п. \ref{sec:doclist});

Текст выпускной квалификационной работы должен быть загружен в систему Антиплагиат через LMS <<НИУ ВШЭ>>.

Документация и программа сдаются в электронном виде в формате .pdf или .docx в архиве формата .zip или .rar;

За две недели до защиты комиссии все материалы курсового проекта:
\begin{itemize}
  \item техническая документация,
  \item программный проект,
  \item исполняемый файл,
  \item отзыв руководителя,
  \item лист Антиплагиата
\end{itemize}
должны быть загружены одним или несколькими архивами в личном кабинете информационной образовательной среды LMS (Learning Management System) НИУ ВШЭ.



\section{Стадии и этапы разработки}
\subsection{Техническое задание}
Обоснование необходимости разработки:
\begin{enumerate}
  \item Определение целей и задач проекта;
  \item Изучение теоретических основ, необходимых для реализации проекта;
  \item Выбор и обоснование критериев эффективности и качества разрабатываемого продукта;
  \item Обоснование необходимости проведения научно-исследовательских работ;
\end{enumerate}
Научно-исследовательские работы:
\begin{enumerate}
  \item Предварительный выбор методов решения поставленной задачи;
  \item Определение синтаксических правил расширенного языка;
  \item Определение требований к техническим средствам;
  \item Выбор подхода к компиляции, а также приоритетной программной среды, в которой будет работать компилятор;
  \item Обоснование возможности решения поставленной задачи.
\end{enumerate}
Разработка и утверждение технического задания:
\begin{enumerate}
  \item Определение требований к программе;
  \item Определение стадий, этапов и сроков разработки программы и документации на неё;
  \item Выбор языка программирования (между C и С++);
  \item Согласование и утверждение технического задания.
\end{enumerate}
Подготовка и передача программы:
\begin{enumerate}
  \item Утверждение даты защиты программного продукта;
  \item Подготовка программы и программной документации для презентации и защиты;
  \item Представление разработанного программного продукта руководителю и получение отзыва;
  \item Загрузка Отчёта в систему Антиплагиат через ЛМС НИУ ВШЭ;
  \item Защита программного продукта (выпускной квалификационной работы) комиссии.
\end{enumerate}
\subsection{Рабочий проект}
Разработка программы:
\begin{enumerate}
  \item Разработка лексического анализатора;
  \item Разработка синтаксического анализатора;
  \item Разработка реализации новых языковых конструкций и конвертации их в язык C;
  \item Написание тестов;
  \item Отладка программы.
\end{enumerate}
Разработка программной документации:
\begin{enumerate}
  \item Разработка программных документов в соответствии с требованиями ЕСПД.
\end{enumerate}
Испытания программы:
\begin{enumerate}
  \item Разработка, согласование и утверждение программы и методики испытаний;
  \item Проведение предварительных приемо-сдаточных испытаний;
  \item Корректировка программы и программной документации по результатам испытаний.
\end{enumerate}
Сроки разработки и исполнители:
Разработка должна закончиться к 30 мая 2023 года.

Исполнитель: Борисов Константин Николаевич, студент группы БПИ196 факультета компьютерных наук НИУ ВШЭ.
\subsection{Внедрение}
Подготовка и защита программного продукта:
\begin{enumerate}
  \item Подготовка программы и документации для защиты;
  \item Утверждение дня защиты программы;
  \item Презентация разработанного программного продукта;
  \item Передача программы и программной документации в архив НИУ ВШЭ.
\end{enumerate}

\section{Порядок контроля и приемки}

Контроль и приемка разработки осуществляются в соответствии с документом «Программа и методика испытаний».
Защита выполненного проекта осуществляется комиссии, состоящей из преподавателей департамента программной инженерии,
в утверждённые приказом декана ФКН сроки.

\section{Технико-экономические показатели}

\subsection{Предполагаемая потребность}

Отсутствие продвинутых возможностей языка C, таких как полиморфизм и перегрузка функций, может быть ограничением в некоторых сценариях программирования, где высоко желательны объектно-ориентированные программные парадигмы.
Однако на многих высокопроизводительных и встроенных системах использование полноценных объектно-ориентированных языков программирования может быть непрактичным или нежелательным из-за дополнительных накладных расходов и сложности, которые они вносят.
Отсутствие этих возможностей в языке C означает, что разработчики должны полагаться на другие техники программирования, чтобы достичь аналогичной функциональности.
Хотя эти техники могут быть эффективными, они также могут быть более громоздкими и подверженными ошибкам по сравнению с встроенными возможностями языка.

\subsection{Преимущества разработки по сравнению с отечественными и зарубежными аналогами}

Процедурно-параметрическая парадигма была реализована в нескольких языках программирования, включая CLOS~\cite{demichiel1989overview} и Oberon-2~\cite{легалов2003процедурный}, которые не так широко используются, как C~\cite{bissyande2013popularity}.
Реализация процедурно-параметрической парадигмы в качестве расширения для C позволит привлечь к этой программной парадигме значительно большую аудиторию.
Кроме того, расширение компилятора C может помочь решить некоторые проблемы, связанные с отсутствием объектно-ориентированных возможностей в C, добавив новые языковые возможности и способности к языку.
Эти расширения могут быть реализованы таким образом, чтобы быть совместимыми с существующим кодом и инструментарием на C, что позволяет разработчикам использовать новые возможности, не полностью отказываясь от C или его экосистемы.
