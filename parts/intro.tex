Разработка компилятора языка программирования C является сложной задачей, которая затрагивает несколько областей исследований: разбор, дизайн языка, семантический анализ и генерация кода. Эта исследовательская работа направлена на разработку нового языка программирования и соответствующего ему компилятора, чтобы удовлетворить потребности определенной программной парадигмы.

Процедурно-параметрическая парадигма~\cite{легалов2000процедурно} --- это программная парадигма, которая объединяет элементы процедурного программирования и параметрического полиморфизма. Параметрический полиморфизм позволяет использовать один и тот же код с разными типами, обеспечивая способ написания более общего и повторно используемого кода.

Возможность интеграции данной парадигмы в язык программирования С уже была изучена в более ранних научных работах~\cite{легалов2016эволюционное}.
За последние годы было предложено много потенциальных способов синтаксической маркировки конструкций процедурно-параметрической парадигмы~\cite{легалов2007применение}.
В связи с чем необходимо было сделать выбор между несколькими вариантными языковой грамматики, в соответствие с удобством её использования, эстетической совместимостью с существующими конструкциями языка программирования С, а также простотой синтаксического анализа.

Написание данного проекта в виде расширения для компилятора C предоставляет возможность для интероперабельности с его экосистемой, включая компиляторы, отладчики и другие инструменты разработки.
Таким образом, разработка расширения позволит использовать новые возможности, предоставленные данной парадигмой, наравне с традиционными функциями и инструментами языка C.
Это позволит сохранить совместимость с существующим кодом на языке C, а также на других компилируемых языках совместимых с данной экосистемой, таких как С++, ассемблер и FORTRAN~\cite{donev2006interoperability}.
В добавок, такой подход позволит сократить затраты на переписывание существующего кода, предоставляя разработчикам новые возможности и более гибкие инструменты для решения задач программирования.


% Стабильность в психологическом состоянии человека важна по ряду причин. Во-первых, это помогает лучше управлять эмоциями, более ясно мыслить и принимать более рациональные решения. Это также может помочь лучше справляться со стрессом \cite{importance} и строить более крепкие отношения с другими людьми. Стабильное психологическое состояние может привести к повышению самооценки и повышению удовлетворенности жизнью.

% Хотя консультация с психологом может быть лучшим методом решения психологических проблем, затраты времени и стоимость таких услуг могут не позволить человеку воспользоваться ими. В этом случае они могут захотеть решить проблему самостоятельно.

% Таким образом, можно использовать самопомощь для поддержания стабильного психологического состояния \cite{self-help}. Это может включать в себя такие действия, как чтение психологического контента \cite{methods} или проведение психологических опросов, приемы релаксации, позитивный разговор с самим собой и т. д. Однако общедоступного контента, касающегося психологии и самопомощи, не хватает. Кроме того, на рынке мобильных устройств в настоящее время нет предложений, сочетающих различные типы материала для самопомощи одновременно.
