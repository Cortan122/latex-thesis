\begin{terminologyList}
  \term[Компилятор (Compiler)]{это программа, которая преобразует исходный код на языке программирования в машинный код для выполнения на целевой платформе}
  \term[Токен (Token)]{это минимальная единица лексического анализа, представляющая собой логически связанный набор символов, например, идентификатор, число или знак операции}
  \term[Лексический анализатор (Lexer)]{разбивает входной код на токены для дальнейшей обработки}
  \term[Синтаксический анализатор (Parser)]{анализирует структуру токенов и проверяет соответствие синтаксису языка}
  \term[Абстрактное синтаксическое дерево (AST)]{представляет собой иерархическую структуру, отражающую синтаксис языка и упрощающую дальнейшую обработку}
  \term[Генератор кода (Emitter)]{преобразует AST в код на целевом языке}
  \term[Таблица символов (Symbol Table)]{хранит информацию о идентификаторах в программе и позволяет обеспечить правильное разрешение ссылок на них}
  \term[Метакомпилятор (Metacompiler)]{это инструмент для автоматической генерации кода на языке С, который анализирует метаданные в программе, и оставляет большую часть кода без изменений}

  \term[Процедурно-параметрическая парадигма (Procedural-Parametric Paradigm)]{это парадигма программирования, которая позволяет создавать параметризованные типы данных и функции, расширяющие базовую функциональность языка программирования}
  \term[Мультиметод (Multimethod)]{это метод, реализация которого зависит от типов нескольких объектов, переданных в качестве аргументов}
  \term[Обобщенные типы]{это типы, которые могут быть параметризованы другими типами и использоваться для создания абстрактных или универсальных структур данных и функций}
  \term[Интероперабельность]{это способность различных систем, программ или устройств взаимодействовать и работать вместе без дополнительных сложностей}
  \term[Экосистема C]{это совокупность связанных инструментов, форматов файлов, библиотек и языковых средств, которые используются в разработке программного обеспечения на языке C и его производных, таких как C++}
  \term[Динамический полиморфизм]{полиморфизм, в котором вызываемый метод определяется на основе реального типа объекта во время выполнения программы}

  \term[Хеш-таблица (Hash Table)]{это структура данных, которая используется для хранения и быстрого поиска пар ``ключ-значение'', где каждый элемент идентифицируется уникальным ключом, который после хеширования преобразуется в индекс в массиве}
  \term[Объединение (Union)]{это тип данных в языках программирования, который позволяет объединять несколько переменных разных типов в одну структуру, где каждая переменная занимает общее пространство памяти}
  \term[Единица компиляции (Translation unit)]{это единица исходного кода, которая обычно включает в себя один или несколько файлов исходного кода, и которая может быть скомпилирована в один объектный файл или статическую библиотеку}
  \term[Компоновка (Linking)]{это процесс объединения нескольких объектных файлов или библиотек в единую исполняемую программу или динамическую библиотеку}
  \term[Конструктор]{функция определённая с использованием специальных атрибутов, которая будет выполнена автоматически до выполнения функции main} % Не стоит путать с конструкторами в ООП
  \term[Динамическая память]{это память, выделяемая во время выполнения программы и используемая для хранения данных, размер которых не может быть определён на этапе компиляции}
\end{terminologyList}
