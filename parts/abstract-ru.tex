% Данный отчет описывает разработку расширения компилятора C, основанного на процедурно-параметрической парадигме, программной парадигме, которая позволяет определять параметрические типы и функции.
% Расширение было реализовано с помощью переводчика исходного кода, написанного на языке C.
% % Отчет будет включать экспериментальную оценку расширения по сравнению с другими подходами к компиляции и синтаксическим вариациям с использованием метрик, таких как время компиляции, размер кода и сложность расширения.
% В целом, расширение предоставляет способ расширения языка C с продвинутыми функциями, сохраняя при этом преимущества производительности языка.
% % Отчет также описывает базовые концепции процедурно-параметрической парадигмы.

Данный отчет описывает разработку расширения компилятора C на основе процедурно-параметрической парадигмы программирования. Расширение позволяет использовать параметрические типы и функции в языке C, предоставляя новые возможности для разработчиков.

Отчет включает в себя подробное описание процедурно-параметрической парадигмы, объясняя ее концепции и принципы работы. Также рассматривается реализация расширения с использованием переводчика исходного кода (метакомпилятора), написанного на языке C.
Важно отметить, что разработанный компилятор работает на платформе GNU+Linux и использует продвинутые функции, такие как конструкторы.

Отчет предназначен для разработчиков и специалистов в области компиляторов, которые интересуются новыми парадигмами программирования и расширением языка C. Он предоставляет детальное описание разработки, а также руководство по использованию разработанного расширения в практических проектах.
В целом, расширение предоставляет способ расширения языка C с продвинутыми функциями, сохраняя при этом преимущества производительности языка.
