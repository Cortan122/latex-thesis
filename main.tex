\documentclass{includes/Thesis}

\newif\ifsmallfile
\usepackage{includes/thesiscommon}

\begin{document}
\CRTsign
\begin{thesis}[
  title = Разработка расширений компилятора C для поддержания процедурно-параметрического программирования,
  year = 2023,
  %
  authorGroup = БПИ196,
  authorName = К. Н. Борисов,
  %
  academicTeacherTitle = {Профессор департамента программной инженерии, Доктор технических наук},
  academicTeacherName = А. И. Легалов,
  academicTeacherFullName = Легалов Александр Иванович,
  %
  isAcademic,
  UDC = 004.4'427,
  %
  keywordsRu = компилятор; язык С; парадигма; процедурно-параметриче\-ская; синтаксис; полиморфизм,
  keywordsEn = compiler; C language; paradigm; procedural-parametric; syntax; polymorphism,
  %
  % useSimpleTitlePage
]
  \startlist{toc}
  \setAbstractResource{parts/abstract-ru}{parts/abstract-en}

  \setTerminologyResource{parts/terminology}

  \setIntroResource{parts/intro}

  % TODO: rename chapter
  % Обзор существующих решений и выбор методов решения задач
  \addChapter{Выбор подхода к решению задачи}{parts/chapter1}
  \addChapter{Принципы разработки и архитектурные особенности}{parts/chapter2}
  \addChapter{Технологии и программная реализация}{parts/chapter3}

  \setConclusionResource{parts/conclusion}

  \makeatletter
  \@ifundefined{CRTDisableAllAdditions}{
    \addAppendix{Техническое задание}{parts/tz}
    \addAppendix{Программа и методика испытаний}{parts/pmi}
    \addAppendix{Руководство оператора}{parts/ro}
    \addAppendix{Текст программы}{parts/tp}
  }{}\makeatother

\end{thesis}
\end{document}
