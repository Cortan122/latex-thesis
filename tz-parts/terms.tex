\section{Список обозначений}

\begin{enumerate}
  \item Компилятор (Compiler) --- это программа, которая преобразует исходный код на языке программирования в машинный код для выполнения на целевой платформе.
  \item Токен (Token) --- это минимальная единица лексического анализа, представляющая собой логически связанный набор символов, например, идентификатор, число или знак операции.
  \item Лексический анализатор (Lexer) --- разбивает входной код на токены для дальнейшей обработки.
  \item Синтаксический анализатор (Parser) --- анализирует структуру токенов и проверяет соответствие синтаксису языка.
  \item Абстрактное синтаксическое дерево (AST) --- представляет собой иерархическую структуру, отражающую синтаксис языка и упрощающую дальнейшую обработку.
  \item Генератор кода (Emitter) --- преобразует AST в код на целевом языке.
  \item Таблица символов (Symbol Table) --- хранит информацию о идентификаторах в программе и позволяет обеспечить правильное разрешение ссылок на них.
  \item Метакомпилятор (Metacompiler) --- это инструмент для автоматической генерации кода на языке C, который анализирует метаданные в программе, и оставляет большую часть кода без изменений.

  \item Процедурно-параметрическая парадигма (Procedural-Parametric Paradigm) --- это парадигма программирования, которая позволяет создавать параметризованные типы данных и функции, расширяющие базовую функциональность языка программирования. % экспериментальная парадигма программирования, позволяющая реализовывать множественный полиморфизм на уровне языка.
  \item Мультиметод (Multimethod) --- это метод, реализация которого зависит от типов нескольких объектов, переданных в качестве аргументов.
  \item Обобщенные типы --- это типы, которые могут быть параметризованы другими типами и использоваться для создания абстрактных или универсальных структур данных и функций.
\end{enumerate}
