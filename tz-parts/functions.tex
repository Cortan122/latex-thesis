Разрабатываемый программный продукт должен:

\begin{enumerate}
  \item Предоставлять поддержку расширенных конструкций языка, основанных на принципах Процедурно-Параметрического программирования~\cite{легалов2016эволюционное}, а именно:
  \begin{enumerate}
    \item Определение параметрического обобщения;
    \begin{grammar}
      <ОбобщеннаяСтруктура> ::= <Структура> `<' <СписокСпециализаций> `>'

      <СписокСпециализаций> ::= <ИмяТипа> `;' | <ИмяТипа> `;' <СписокСпециализаций>
    \end{grammar}
    \item Расширение обобщённой структуры;
    \begin{grammar}
      <РасширениеТипа> ::= <ИмяСтруктуры> `+' `<' <СписокСпециализаций> `>'
    \end{grammar}
    \item Использование конкретной специализации, в качестве типа переменных;
    \begin{grammar}
      <Специализация> ::= <ИмяСтруктуры> `<' <ИмяТипа> `>'
    \end{grammar}
    \item Определение специализированной функции;
    \begin{grammar}
      <СпециализированнаяФункция> ::= <ТипВозврата> <ИмяФункции> \\ `<' <СписокОбобщённыхПараметров> `>' `(' <СписокПараметров> `)'
    \end{grammar}
    \item Вызов специализированной функции;
    \begin{grammar}
      <ВызовПараметрическойФункции> ::= <ИмяФункции> \\ `<' <СписокОбобщённыхПараметров> `>' `(' <СписокПараметров> `)'
    \end{grammar}
  \end{enumerate}

  \item Предоставлять поддержку базовых конструкций языка программирования C, таких как переменные, функции, условные операторы и циклы, пропуская код без изменений;
  \item Предоставлять возможность объявления пользовательских типов данных и структур данных;
  \item Предоставлять возможность компиляции одной программы используя несколько единиц компиляции;
  \item Предоставлять возможность обработки исходного кода на языке C, после того как он прошёл этап препроцессора;
  \item Предоставлять возможность получения на выходе кода с правильными номерами строк, которые контролируются особыми директивами препроцессора;

  \item Запускать компилятор и препроцессор языка C и организовывать межпроцессную коммуникацию между ними;
  \item Хранить таблицу имён переопределённых типов, обобщённых структур, специализированных переменных и параметрических функций.
\end{enumerate}
